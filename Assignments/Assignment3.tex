\documentclass[11pt, oneside]{article}   	% use "amsart" instead of "article" for AMSLaTeX format
\usepackage[top=2in, bottom=1.5in, left=1in, right=1in]{geometry}                		% See geometry.pdf to learn the layout options. There are lots.
\usepackage{hyperref}
\usepackage{exercise}

\title{Assignment 3 \\ \large(10\%)}
\author{Instructor: Marie Auger-M\'eth\'e}
\date{}							% Activate to display a given date or no date

\begin{document}
\maketitle

Please send me before next class an e-mail with all the answers for this assignment. All \texttt{R} code should be in one script. Please indicate which lines are associated with each exercise by including comments in your code. In addition, send me a copy of the figures in your e-mail. You can save the figures in .pdf or .jpeg format.

\begin{Exercise}

Create one \texttt{SpatialPolgons} object identifying the study area from the Deployments.csv file from the Ocean Tracking Network (OTN) Arctic Cumberland Sound Array project data found at \url{http://members.oceantrack.org/data/discovery/ACS.htm}. This is the same file you used in assignment 1.

\Question
Import the deployment.csv  and create a \texttt{SpatialPointsDataFrame} with it.

\Question Use the \texttt{bbox} function to  identify the spatial extent of this \texttt{SpatialPointsDataFrame} and create a matrix or data frame with the coordinates of the study area. See tutorial for an example on how to do this.

\Question Use these coordinates to create a \texttt{SpatialPolygons}.

\end{Exercise}

\begin{Exercise}

\Question Plot the \texttt{SpatialPolygons} create in the previous exercise. Plot the axes and make sure the y-axis is horizontal.

\Question Add a map of Canada (again see assignment 1). 

\Question Add \texttt{SpatialPointsDataFrame} created in the previous exercise.

\Question Add a North arrow.

\end{Exercise}
\end{document}  