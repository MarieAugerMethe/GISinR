\documentclass[11pt, oneside]{article}   	% use "amsart" instead of "article" for AMSLaTeX format
\usepackage[top=2in, bottom=1.5in, left=1in, right=1in]{geometry}                		% See geometry.pdf to learn the layout options. There are lots.
\usepackage{hyperref}
\usepackage{exercise}

\title{Assignment 1}
\author{Marie Auger-M\'eth\'e}
\date{}							% Activate to display a given date or no date

\begin{document}
\maketitle

Please send me before next class (Feb 5) an e-mail with all the answers for this assignment. All \texttt{R} code should be in one script. Please indicate which lines are associated with each exercise by including comments in your code.

\begin{Exercise}
One of the advantages of R, is that it has a vibrant user community, with many messages boards and other venues to ask for help and discuss problems. For example, the R-sig-Geo list is a specialise mailing list for R functions that handles spatial data (\url{https://stat.ethz.ch/mailman/listinfo/r-sig-geo}). My favourite way to ask for help with R issues is through Stack Overflow (\url{http://stackoverflow.com/}). I want you to create yourself a Stack Overflow account (\url{https://stackoverflow.com/users/signup}).
Send me by e-mail your stackoverflow username.
\end{Exercise}

\begin{Exercise}
Once signed in with your username on Stack Overflow, read the tour page (\href{http://stackoverflow.com/tour}{stackoverflow.com/tour}). You need to be signed in for me to know whether you have read the tour page, so make sure you have signed in before reading it. 
\end{Exercise}

\begin{Exercise}
Import your own point data into \texttt{R} . If you don't have point data at hand, use the Deployments.csv file from the Ocean Tracking Network (OTN) Arctic Cumberland Sound Array project data found at \url{http://members.oceantrack.org/data/discovery/ACS.htm}. 

\Question Use this data to create a \texttt{SpatialPointsDataFrame}. Make sure the \texttt{Spatial} object has appropriate coordinate reference system (CRS) associated with it. 

\Question Change the CRS of the object created to a projected CRS that uses a conic projection of your choice. 

\Question Make a plot with two panels that show the object in it's original CRS and in it's new projection. Make sure the panels have titles. Overlay on each panel a map showing the nearby landmasses/countries. Use the maps available in the \texttt{maps} package. In addition to the \texttt{R} code, send me a copy of the figure in your e-mail. You can save the figure in .pdf or .jpeg format. 
\end{Exercise}

\end{document}  